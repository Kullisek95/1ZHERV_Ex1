%%%%%%%%%%%%%%%%%%%%%%%%%%%%%%%%%%%%%%%%%%%%%%%%%%%%%%%%%%%%%%%%%%%%%%%%%%%%%%%%
% Author : Jan Kvapil, Tomas Polasek (template)
% Description : First exercise in the Introduction to Game Development course.
%   It deals with an analysis of a selected title from the point of its genre, 
%   style, and mechanics.
%%%%%%%%%%%%%%%%%%%%%%%%%%%%%%%%%%%%%%%%%%%%%%%%%%%%%%%%%%%%%%%%%%%%%%%%%%%%%%%%

\documentclass[a4paper,10pt,english]{article}

\usepackage[left=2.50cm,right=2.50cm,top=1.50cm,bottom=2.50cm]{geometry}
\usepackage[utf8]{inputenc}
\usepackage{hyperref}
\hypersetup{colorlinks=true, urlcolor=blue}

\newcommand{\ph}[1]{\textit{[#1]}}

\title{%
Analysis of Mechanics%
}
\author{%
Jan Kvapil (xkvapi19)%
}
\date{}

\begin{document}

\maketitle
\thispagestyle{empty}

{%
\large

\begin{itemize}

\item[] \textbf{Title:} Stardew Valley

\item[] \textbf{Released:} 26th of February 2016

\item[] \textbf{Author:} Eric Barone / ConcernedApe

\item[] \textbf{Primary Genre:} Farm life simulator

\item[] \textbf{Secondary Genre:} RPG

\item[] \textbf{Style:} Pixel-Art

\end{itemize}

}

\section*{\centering Analysis}

% \subsection*{About the game}

% Stardew Valley is an independent farm life simulator developed solely by Eric Barone (ConcernedApe). The game was originally released in 2016 after over 4 years of development. The newest major update (1.6) was released in March 2024 for PC, and it is speculated that version 1.6 is the last major update to the game. The console and mobile versions have yet to receive this update.

%\subsection*{Gameplay}

Stardew Valley is a farm life simulator. The player, originally an employee at a soulless corporate, decides to abandon their old life full of stress, and finally open an envelope that their grandfather left them on their deathbed. This letter informs the player that they are inheriting their grandfathers old farm. The player journeys into Stardew Valley, a small community of roughly 30 citizens, where they start a new, simple life on a farm. Here, the player is given complete autonomy. They start off by clearing off the farm that has now become overgrown, before eventually planting their first batch of parsnips and beginning their new life. 

% As the player arrives, they are greeted by the mayor of the town, Lewis, and the towns carpenter, Robin. The two welcome the player to the town. After that, the player is given near complete freedom. They can start clearing out their now overgrown farm and start growing their first crops, or they can explore the town and the areas it has to offer.

\subsection*{Primary genre}

As mentioned earlier, Stardew Valley is a farm life simulator. The game fully revolves around this aspect. Over 40 different crops are available in the game. These crops differ in profit margin, length of the growing process, and the season during which they can grow. The player is truly "living off the land", as they plant, water, and harvest different crops, before further processing them, or selling them for profit right away. Similar to real life, later in the game, the player can purchase (or craft using obtained materials) items such as sprinklers, which automatically water the plants, and scarecrows, which scare away crop-eating birds.

% Seasons are an important aspect of the game, too. Similar to real life, different crops thrive during different seasons. The most profitable crops grow during the summer, while spring and fall crops are less profitable, and can be used to make artisan goods, such as wine, jelly, and pickles.

Another farming aspect of the game is husbandry. The player can have barns and coops built by Robin (the towns carpenter), and then house farm animals such as cows, goats, chickens, ducks, and many more in them. These animals are purchased from Marnie, another villager in the town. These animals can be milked, sheared, and petted to improve their happiness and production. Similarly to plants, there are machines (such as the "Auto-Grabber" or the "Auto-Petter") that perform some tasks automatically available in mid/late-game.

\subsection*{Secondary genre}

The player is part of the community in the game. It is possible to interact with townspeople, to discover their backstories, and to grow closer with them. By talking with villagers, giving them gifts, or assisting them, the player can earn "friendship points". Upon reaching milestones, special, one-time cut-scenes play, often containing decisions to be made by the player. These decisions can further affect the friendship with the specific NPC. There are twelve NPCs that the player can marry. Six of them are male, six of them are female.

\subsection*{Style}

The game has a pixel-art style, with all aspects of the game being pixel-art at different levels of detail and pixel density. This choice, together with an amazing soundtrack, works well in creating a cozy, welcoming atmosphere, and fully immerse the player into the game. Both the art and the soundtrack for the game are masterfully crafted by ConcernedApe himself. 





%\subsection*{Instructions}

%In this assignment, you are tasked with the analysis of a selected game-related title. The title may be a game, video game, serious game, or even serious application using game development tools. Your goal is to analyze the title from the point of its genres and style. As a part of this template, there are some placeholders and hints \ph{like this one}, which you should read and potentially replace with your own text.

%\subsection*{Content}

%After selecting the \ph{title}, you should first find out when it was \ph{first released} and who \ph{created it}. Be sure to consider the actual information if you choose a re-iteration or ``enhanced edition.'' 

%Next, look at the game (or, even better, play it!) and determine the \ph{primary genre}. This genre should be the one supporting the core gameplay. You can use any genre taxonomy (not just the one from the lectures), but keep it unambiguous. A Game can have multiple modes of play -- e.g., Minecraft with creative and survival modes -- in which case you can choose any number of them, but be sure to emphasize your choice in the analysis.

%After these steps, look at the \ph{secondary genres} and select one or more of them. Using Survival Minecraft as an example, we have a role-playing sandbox (primary) combined with the casual building and a hint of roguelike with the hardcore mode (secondary). Finally, determine the game's \ph{style} -- a combination of visual, aural, tactile, etc. For example, Minecraft can be considered a retro or cartoon-styled game.

%Finally, move to the \ph{free-form text} part of the analysis in the form of short prose. Images should be used sparingly and best avoided them entirely. You should primarily focus on: 
%\begin{enumerate}
    %\item How are the primary and secondary genres reflected in the gameplay?
    %\item How do the primary and secondary genre interact? Do the secondary genres support the primary genre? Do they enhance the game, or are they detrimental?
    %\item Does the style support the gameplay? Why was it chosen?
%\end{enumerate}

%\subsection*{Formatting \& Submission}

%Your submission must follow a similar \textbf{structure} as this template. You can either use the provided \LaTeX\ template or roughly replicate it in some other text processing software. The format of the analysis section is left up to you -- you can include sub-sections or write one long text. However, your whole document \textbf{must fit} on exactly one page of \textbf{A4}. The only accepted document format is \textbf{pdf}. Finally, you can submit the pdf by following the submission guidelines detailed on the \href{http://cphoto.fit.vutbr.cz/ludo/courses/izhv/exercises/sub/}{course's website}.

\end{document}
